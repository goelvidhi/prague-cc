% !TeX root = responsiveness_tr
%% ================================================================
\section{Approximations}\label{prresp_approx}

The per-ACK EWMA is not intended to precisely mimic a per-RTT EWMA. Otherwise, the per-ACK EWMA would have to reach the same value by the end of the round, irrespective of whether markings arrived early or late in the round.
It is more important for the EWMA to quickly accumulate any markings early in the round than it is to ensure that the EWMA reaches the same value by the end of the round. 

It is not important that a per-ACK EWMA decays at the same rate as a per-round EWMA using the same gain. The gain is not precisely chosen, so if a per-ACK EWMA decays more slowly, a higher gain value can be used for a per-ACK algorithm.

It \emph{is} important that a per-ACK EWMA decays at about the same rate however many ACKs there are per round, although the decay rate does not have to be precisely the same.

The per-ACK approach uses the approximation that one reduction with gain \(1/g\) is roughly equivalent to \(n\) repeated reductions with \(1/n\) of the gain. Specifically, that \((1 - 1/ng)^n \approx 1 - 1/g\), for \(g\ge2; n\ge2\).

\begin{align*}
(1 - 1/ng)^n &=       1 + \frac{n}{-ng} + \frac{(n-1)}{(-ng)^2} + \ldots \\
             &=       1 - \frac{1}{g} + O\left(\frac{1}{n(-g)^2}\right)\\
             &\approx 1 - \frac{1}{g}
\end{align*}
.
